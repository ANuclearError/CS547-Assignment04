\documentclass[11pt, a4paper]{article}
\usepackage[parfill]{parskip}
\usepackage[margin=1in]{geometry}
\usepackage{url}
\usepackage{listings}
\usepackage{graphicx}
\usepackage{float}

\lstdefinestyle{snippet}{
    numbers = left,
    basicstyle = \ttfamily\footnotesize,
    frame = single,
    breaklines = true,
}

\setlength\parindent{0pt}

\begin{document}
\title{CS547 Advanced Topics in Computer Science\\
\large{Assignment 04 - Genetic Programming: Software Cost Estimation}}
\author{Aidan O'Grady - 201218150}
\date{}
\maketitle
\section{Introduction} % (fold)
\label{sec:introduction}
The purpose of this assignment was to utilise genetic programming in approaching
the problem of software cost estimation. When a software project is being
proposed, there are multiple factors that can influence the eventual cost of the
software. It can be difficult to determine which factors are more important to
consider, making this an appropriate search challenge in the scope of genetic
programming.

Using datasets provided, the primary goal was to identify which factors should
be taken into account to achieve the closest possible prediction to the cost
outlined in the file. Using various metrics, these estimations will be compared
to other methods that can be used to estimate costs such as linear regression.
% section introduction (end)

\section{Background} % (fold)
\label{sec:background}
There were around 15 .arff files available for use as part of this assignment.
Of those, I settled on using the \emph{Albrecht, Kemerer} \& \emoh{Miyazaki}
datasets as part of my work. These three were chosen due to their relative
straightforwardness compared to other files. These datasets were also used by
Dolado \cite{Dolado200161} in a related study, which included results obtained
from his own genetic programming experimentation (providing me with references
to help understand the datasets). Knowing that I could use his results as a
reference point helped in ensuring that I stayed on the right track with this
problem.

\subsection{Albrecht} % (fold)
\label{sub:albrecht}
With the Albrecht dataset, I was able to understand the `function point' concept
used in various datasets. The original paper by Albrecht \& Gaffney
\cite{1703110}, showed how the \emph{RawFPcounts} attribute was calculated
from the \emph{Input, Output, Inquiry \& File} attributes, with a multiplier
then provided as well to add a modified value as well. With this dataset, I
decided to ignore the adjusted figure, instead opting to include the raw
function point value and its multiplier. I also opted to include the attributes
that make up the function points value, in case that the function points alone
could not provide the most accurate prediction.

Included: \emph{Input, Output, Inquiry, File, FPAdj, RawFPcounts}

Excluded: \emph{AdjFP}

Effort: \emph{Effort}
% subsection albrecht (end)

\subsection{Kemerer} % (fold)
\label{sub:kemerer}
The Kemerer dataset also included both raw and adjusted function point counts,
so the adjusted value was again ignored for the sake of consistency. The
\emph{ID} was ignored due to only being an identifier.

This dataset also included two attributes, \emph{Language} \& \emph{Hardware},
that acted as enumerations. This had to be addressed, since while these are going
to be factors in the real world, I was not certain whether I could trust the
enumerated values to be reliable in providing cost estimations. Thus, these were
also excluded. \emph{Duration} \emph{KSLOC} were difficult to decide on, due to
uncertainty as to their appropriateness. Eventually, I decided to exclude them,
due to this uncertainty.

Included: \emph{RAWFP}

Excluded: \emph{ID, Language, Hardware, Duration, KSLOC, AdjFP}

Effort: \emph{EffortMM}
% subsection kemerer (end)

\subsection{Miyazaki} % (fold)
\label{sub:miyazaki}
As with \emph{Kemerer}, the \emph{ID} and \emph{KLOC} attributes were excluded
for the same reasons. This meant that the only real decision to be made with
this set is which group of three attributes to choose. From reading the original
source \cite[Section~4.2]{MIYAZAKI19943}, I decided it was best to try with both
and choose whichever performed better with genetic programming. This became
\emph{SCRN, FORM} and \emph{FILE}. It may have been worth investigating some
combinations between them as well however.

Included: \emph{SCRN, FORM, FILE}

Excluded: \emph{ID, KLOC, ESCRN, EFORM, EFILE}

Effort: \emph{MM}
% subsection miyazaki (end)
% section background (end)

\section{Implementation Details} % (fold)
\label{sec:implementation_details}
% section implementation_details (end)

\section{Results} % (fold)
\label{sec:results}
% section results (end)

\section{Comparisons} % (fold)
\label{sec:comparisons}
% section comparisons (end)
\bibliographystyle{plain}
\bibliography{report} 
\end{document}